\documentclass[a4paper,12pt]{article}
\usepackage[T1]{fontenc}
\usepackage[utf8]{inputenc}
\usepackage{url}
\usepackage{fancyhdr}
\usepackage{lastpage}
\usepackage{hyperref}

\setlength\parindent{0pt}

\pagestyle{fancy}
\fancyhf{}
\lhead{Codeforces notes}
\cfoot{\thepage/\pageref{LastPage}}

\title{Codeforces notes}
\author{Mario Ynocente Castro (MarioYC)}
\date{}

\begin{document}

\maketitle
\thispagestyle{empty}

\newpage
\tableofcontents

\newpage
\section{\href{http://codeforces.com/contest/687}{Codeforces Round 360 (Div 1)}}

\subsection{Dividing Kingdom II}

In any interval sort the edges in decreasing order by their weight, then the important edges are the ones that join two different components and the first edge that joins vertices that are in the same component and apart from each other by an even number of edges, so the number of important edges is at most n.
\\ \\
We can build a segment tree that keeps the important edges for an interval, and then for each query we will do at most $\log n$ steps that consist of merging these sets of important edges.
\\ \\
\textbf{Implementation:} \href{http://codeforces.com/contest/687/submission/19196725}{C++ code}
\\ \\
\textbf{Notes:} The same approach would work for similar problems like MST and number of connected components.

\section{\href{http://codeforces.com/contest/691}{Educational Codeforces Round 14}}

\subsection{Couple Cover}

If all $a_i$ are different and $P = 3 \cdot 10^6$ then the number of pairs $(i,j)$ for which $a_i a_j \leq P$ is $O(P\log P)$ and if these values are sorted then we can go through these pairs in $O(n + P\log P)$. For the general case, we will first need to compress these values.
\\ \\
While going through the values we can increase a counter for each different value of the product, then we can accumulate these counters.
\\ \\
\textbf{Implementation:} \href{http://codeforces.com/contest/691/submission/19352079}{C++ code}

\end{document}